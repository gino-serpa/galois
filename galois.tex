\documentclass[10pt,a4paper]{article}
\usepackage[utf8]{inputenc}
\usepackage{amsmath}
\usepackage{amsfonts}
\usepackage{amssymb}
\begin{document}

\section{LINEAR ALGEBRA}

\subsection{Fields}

A field is a set of elements in which a pair of operations called
multiplication and addition is defined analogous to the operations of
multiplication and addition in the real number system (which is itself
an example of a field). In each field $F$ there exist unique elements 
called $0$ and $1$ which, under the operations of addition and multiplication, behave with respect to all the other elements of $F$ exactly as their 
correspondents in the real number system. In two respects, the
analogy is not complete: 1) multiplication is not assumed to be 
commutative in every field, and 2) a field may have only a finite number
of elements.

More exactly, a field is a set of elements which, under the above
mentioned operation of addition, forms an additive abelian group and
for which the elements, exclusive of zero, form a multiplicative group
and, finally, in which the two group operations are connected by the
distributive law. Furthermore, the product of $0$ and any element is defined to be $0$.

If multiplication in the field is commutative, then the field is called a commutative field.

\subsection{Vector Spaces}

If $V$ is an additive abelian group with elements $A, B, . . . , F$ a field with elements 
$a, b, . . . ,$ and if for each $a \in F$ and $A \in V$ the product $aA$ denotes an element 
of $V$, then $V$ is called a (left) vector space over $F$ if the following assumptions hold:

\begin{itemize}
\item $a(A + B) = aA + aB$
\item $(a + b)A = aA + bA$
\item $a(bA) = (ab)A$
\item $1A = A$
\end{itemize}


The reader may readily verify that if $V$ is a vector space over F, then
$oA = o$ and $a0 = 0$ where $o$ is the zero element of $F$ and $0$ that of $V$.
For example, the first relation follows from the equations:

\[
aA = (a + o)A = aA + oA
\]

Sometimes products between elements of $F$ and $V$ are written in
the form $Aa$ in which case $V$ is called a right vector space over $F$ to
distinguish it from the previous case where multiplication by field elements 
is from the left. If, in the discussion, left and right vector
spaces do not occur simultaneously, we shall simply use the term
“vector space”.

\subsection{Homogeneous Linear Equations}

If in a field $F$, $a_{ij}$, $i = 1,2,. . . , m$, $j = 1,2, . . . ,$ $n$ are $mn$ 
elements, it is frequently necessary to know conditions guaranteeing the
existence of elements in $F$ such that the following equations are satisfied:

\[
a,, xi + a,, x2 + . . . + alnxn = 0.
(1) . *
a ml~l + amzx2 + . . . + amnxn = 0.
\]

The reader Will recall that such equations are called linear
homogeneous equations, and a set of elements, xi, x2,. . . , xr,
of F, for which a11 the above equations are true, is called

3
a solution of the system. If not a11 of the elements xi, xg, . . . , xn
are o the solution is called non-trivial; otherwise, it is called trivial.
THEOREM 1. A system of linear homogeneous equations always

has a non-trivial solution if the number of unknowns exceeds the num-
ber of equations.

The proof of this follows the method familiar to most high school
students, namely, successive elimination of unknowns. If no equations
in n > 0 variables are prescribed, then our unknowns are unrestricted
and we may set them a11 = 1.
We shall proceed by complete induction. Let us suppose that
each system of k equations in more than k unknowns has a non-trivial
solution when k < m. In the system of equations (1) we assume that
n > m, and denote the expression a,ixi + . . . + ainxn by L,, i = 1,2,. . .,m.
We seek elements xi, . . . , x,, not a11 o such that L, = L, = . . . = Lm = o.
If aij = o for each i and j, then any choice of xi , . . . , xr, Will serve as
a solution. If not a11 aij are o, then we may assume that ail f o, for
the order in which the equations are written or in which the unknowns
are numbered has no influence on the existence or non-existence of a
simultaneous solution. We cari find a non-trivial solution to our given
system of equations, if and only if we cari find a non-trivial solution
to the following system:
L, = 0
L, - a,,a,;lL, = 0
. . . . .
Lm - amia,;lL, = 0

For, if xi,. . . , x,, is a solution of these latter equations then, since
L, = o, the second term in each of the remaining equations is o and,
hence, L, = L, = . . . = Lm = o. Conversely, if (1) is satisfied, then
the new system is clearly satisfied. The reader Will notice that the
new system was set up in such a way as to “eliminate” x1 from the
last m-l equations. Furthermore, if a non-trivial solution of the last
m-l equations, when viewed as equations in x2, . . . , xn, exists then
taking xi = - ai;‘( ai2xz + ar3x3 + . . . + alnxn) would give us a
solution to the whole system. However, the last m-l equations have
a solution by our inductive assumption, from which the theorem follows.
Remark: If the linear homogeneous equations had been written
in the form xxjaij = o, j = 1,2, . . . , n, the above theorem would still
hold and with the same proof although with the order in which terms
are written changed in a few instances.
D. Dependence and Independence of Vectors.
In a vector space V over a field F, the vectors A,, . . . , An are
called dependent if there exist elements xi, . . . , x”, not a11 o, of F such
that xiA, + x2A, + . . . + xnAn = 0. If the vectors A,, . . . ,An are
not dependent, they are called independent.
The dimension of a vector space V over a field F is the maximum
number of independent elements in V. Thus, the dimension of V is n if
there are n independent elements in V, but no set of more than n
independent elements.
A system A,, . . . , A, of elements in V is called a
generating system of V if each element A of V cari be expressed

5
linearly in terms of A,, . . . , Am, i.e., A = Ca.A. for a suitable choice i=ll 1
ofa,, i = l,..., m,inF.
\end{document}